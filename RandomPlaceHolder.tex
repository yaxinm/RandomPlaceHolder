\documentclass[12pt]{article}
\usepackage{preamble}
\usepackage{placeholder_init/randomplaceholder}
\placeholderclean%
\placeholderinit{placeholderimage}

\title{randomplaceholder user manual}
\author{yaxinm}
\date{September 2022}

\begin{document}
    \maketitle
    Want to add bit of fun to your writing process?
    Insert a random image into your article by typing \lstinline{\placeholder}. 
    \placeholder[@][0.3\textwidth][A random figure.]
    \section{Supported platforms}
    \textbf{Unix} systems and \textbf{Overleaf}, compiled with pdfLaTex. 
    The package (not really a package, but I will call it as such) 
    has been tested on Ubuntu 20.04 (x86\textunderscore64) and Overleaf.

    Sadly, it doesn't work on \textbf{Windows} yet.

    \section{Installation}
    Installation is the same for local \LaTeX~project or Overleaf.

    \subsection{Dependencies}
    Make sure you include the following packages in your \lstinline{preamble.sty}, or at the top of your \lstinline{main.tex}.
    \begin{lstlisting}
        lcg
        graphicx
        xargs
        float
    \end{lstlisting}

    \subsection{Add package to your project}
    The folder \lstinline{placeholder_init} contains all source codes.
    Copy it into your project, place it right next to your \lstinline{main.tex}. 
    
    Then add this line to the top of your file.
    \begin{lstlisting}
        \usepackage{placeholder_init/randomplaceholder}
    \end{lstlisting}

    Now, prepare a folder full of fun images. Make sure that folder contains only images (any format) and the names of both the directory and the files follow the Unix naming conventions. For example, \lstinline{placeholderimage/} in this repo.

    \section{How to use}
    \subsection{Initialise}
    Initialise the package by adding the following command at the top of your file.
    \begin{lstlisting}        
        \placeholderinit{$YourImagefolder}
    \end{lstlisting}
    By running \lstinline{\placeholderinit}, a file containing the macros will be created base on the content of \lstinline{$YourImageFolder}. 
    Once the macro file is present, \lstinline{\placeholderinit} will not overwrite it even if the content of \lstinline{$YourImageFolder} has changed or if you recompile. 
    To reinitiate, call \lstinline{\placeholderclean} before calling \lstinline{\placeholderinit}.

    \subsection{Add a placeholder}
    Use \lstinline{\placeholder} to add a image randomly selected from \lstinline{$YourImageFolder}, no argument necessary. 
    For more control, see next section.

    \section{Available commands}
    \begin{description}
        \item[\lstinline!\placeholderinit{#1}!] 
        Initialise the randomplaceholder by creating a \linebreak \lstinline!placeholdermacros.tex! in the source directory. The command is skipped over if the macro file already exist.
            \begin{itemize}
                \item[\lstinline!#1!] path to the folder with the space holder images. Only images should be present in the folder and names should follow the Unix conventions.
            \end{itemize}
        \item[\lstinline!\placeholderclean!] Remove \lstinline!placeholdermacros.tex! so the package can be reinitialised to reflect changes 
        \item[\lstinline!\placeholder[#1][#2][#3][#4]!] Insert a randomly selected image. Use \lstinline{@} to use default values.
            \begin{itemize}
                \item[\lstinline{#1}] Optional, default=htb. Position of the figure, takes floating figure position specifiers.
                \item[\lstinline{#2}] Optional, default=\lstinline{\textwidth}. Takes a width specifier.
                \item[\lstinline{#3}] Optional, default=None. Text in the figure caption.
                \item[\lstinline{#4}] Optional, default=None. Text in the figure label.  
            \end{itemize}
        \item[\lstinline!\includeplaceholder[#1]!] Use this in place of \lstinline!\includegraphics! in the \lstinline{figure} environment.
            \begin{itemize}
                \item[\lstinline{#1}] Optional, default=\lstinline{width=\textwidth}. Take key-value list for command \lstinline{\includegraphics} from the \lstinline{graphicx} package.
            \end{itemize}
        \item[\lstinline!\placeholderimage{#1}!] Returns the name of the \lstinline{#1}\textsuperscript{th} image in the image folder. 
    \end{description}

    \section{Examples}
    Here are some codes and their results
    \begin{lstlisting}
        \placeholder
        \placeholder[htb][5cm]
        \placeholder[@][0.3\textwidth][The third argument is the caption]
        \placeholder[@][0.3\textwidth][@][fig5]
        Reference figure 5 with \ref{fig5} 
        
        % or use a image in your own \begin{figure}
        \begin{figure}
            \centering
            \includeplaceholder[width=5cm]
            \caption{Use `includeplaceholder' for more control.} 
        \end{figure}
    \end{lstlisting}
    \placeholder%
    \placeholder[htb][5cm]
    \placeholder[@][0.3\textwidth][The third argument is the caption]
    \placeholder[@][0.5\textwidth][@][fig5]
    Reference figure 5 as~\ref{fig5} 
    \begin{figure}
        \centering
        \includegraphics[width=0.3\textwidth]{\placeholderimage{2}}
    \end{figure}
    \begin{figure}
        \centering
        \includeplaceholder[width=5cm]
        \caption{Use `includeplaceholder' for more control.} 
    \end{figure}
\end{document}